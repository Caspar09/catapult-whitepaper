\refstepcounter{section}
\section*{前言}
\addcontentsline{toc}{section}{前言}

\nemquote{
你错过了100\%的你没有尝试的击球机会。
}{韦恩·格雷茨基\footnote{韦恩·格雷茨基,加拿大的职业冰球明星。}}

NEM最初的起源是在2014年1月bitcointalk论坛中一篇帖子里的“投名状”。
加密数字货币领域刚刚经历了2013年底的一波爆发,尽管当时的价格与数年之后相去甚远,可这时的人们总是充满了热情。
NXT主网的发布使得其成为了最早期的POS区块链之一,NEM的早期社区从中汲取了灵感和人才,其中就包括了三位至今仍活跃着的核心研发者。

最初有很多关于我们要做什么的讨论,但我们很快决定要从零开始做一些全新的东西。
This allowed for more design flexibility as well as the use of high coding standards from the very beginning.
This also gave us the opportunity to contribute something new to the blockchain landscape.
我们想要挑战自己,看看能不能做出什么有用的产品来。
作为很多个周末跟夜晚辛勤付出的结果,我们终于在2015年的3月份发布了名为NIS1的NEM主网。
我们对于这个产品感到很满意,但是我们清楚在开发的过程中也采取了一些捷径,于是我们决定继续改善它。
最终,我们意识到了原先解决方案需要一次技术重构来突破核心性能瓶颈,同时适应未来更快速的创新。

我们非常感激TechBureau对我们从零开始创建一条全新的链的支持 - \codename.
我们希望这些条新的链可以解决NIS1存在的众多问题,并且为未来的性能增强和发展提供坚实的基础。
我们的使命是创建一个高性能的“区块链”,而非基于DAG或是dBFT的系统。
在这种意义上而言,我们认为,我们是成功的。

这对于我们来说是一个漫长的旅程,我们希望这是最后一条我们从零开始构建的区块链。
This has been a long journey for us, but we are excited to see what yet is to come and what novel things \textbf{you} use \codenamespace to build.
我们对于未来感到兴奋,也期待看到你们使用\codenamespace创建的全新产品。
最后,再次感谢给予那些我们灵感和支持的人们\ldots

\begin{flushright}
BloodyRookie
gimre
Jaguar0625
\end{flushright}
