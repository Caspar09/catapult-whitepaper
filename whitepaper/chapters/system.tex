\section{系统}
\label{sec:system}

\nemquote{
}{}


\codenamespace在网络级别和单个节点级别都支持高度的自定义。
\nemsetting {network} {network}中指定的网络范围设置对于网络中的所有节点必须相同。
\nemsetting {node} {node}中指定的特定于节点的设置在同一网络中的各个节点之间可能会有所不同。
\codenamespace was designed to use a \emph{plugin / extension} approach instead of supporting Turing complete smart contracts.
While the latter can allow for more user flexibility, it's also more error prone from the user perspective.
A plugin model limits the operations that can be performed on a blockchain and consequently has a smaller attack surface.
Additionally, it's much simplier to optimize the performance of a discrete set of operations than an infinite set.
This helps \codenamespace achieve the high throughput for which it was designed.

\subsection{事务插件}


网络中的所有节点必须支持相同类型的事务,并以完全相同的方式处理它们,以便所有节点可以就全局区块链状态达成共识。
网络事务由一组网络需要每个节点都需要加载的\emph{transaction plugins}\index{plugins}来支持。
这个(事务插件)组是由存在\nemsetting{network}{network} 配置部分。
这些插件的任何更改,添加或删除必须由所有网络节点协调并接受。
如果仅节点的部分子集同意这些修改,则这些节点将在分叉上。
\codenamespace所有内置的事务都使用此插件模型构建,以验证其可扩展性。

插件是一个单独的动态链接库,它以以下形式开放单个接口
\footnote{插件的格式取决于目标操作系统和使用的编译器,因此所有主机应用程序和插件必须使用相同的编译器版本和选项构建。}:

\begin{lstlisting}
extern "C" PLUGIN_API
void RegisterSubsystem(
	catapult::plugins::PluginManager& manager);
\end{lstlisting}

\class{PluginManager} 类提供对插件初始化所需的配置子集的访问。
通过这个类,插件可以注册以下零个或多个:

\begin{enumerate}
	\item{事务(Transaction) -
		可以指定新的事务类型以及这些类型到解析规则的映射。
		具体来说,该插件定义了将交易转换为组件通知的规则,这些通知将在进一步处理中使用。
		还可以指定一些处理约束,例如指示事务只能出现在聚合事务中\nemrefparens{sec:transactions:aggregate}。
	}
	\item{缓存(Caches) -
		可以指定新的缓存类型和规则,以将模型类型与二进制进行序列化和反序列化。
		启用该功能后,可以选择将每个与状态相关的缓存包括在块的\field{StateHash}\nemrefparens{sec:blocks:statehash}的计算中。
	}
	\item{联络程序(Handlers) - 总是可以访问的APIs。}
	\item{诊断器(Diagnostics) - 仅当节点以诊断方式运行时才可访问的API和计数器。}
	\item{验证器(Validators) -
		无状态和有状态验证器,用于处理由区块和事务处理产生的通知。
		经过注册的验证器可以订阅常规或插件定义的通知,并拒绝不允许的值或状态更改。	
	}
	\item{监察器(Observers) -
		监察器处理由区块和事务处理产生的通知。
		经过注册的监察器可以订阅常规或插件定义的通知,并根据其值更新区块链状态。
		监察器不需要任何验证逻辑,因为只有在所有适用的验证者成功之后才调用监察器。
	}
	\item{解析器(Resolvers) -
		可以指定从未解析类型到已解析类型的自定义映射。
		例如,名称空间插件使用它来添加对别名(Alias)解析的支持。
	}
\end{enumerate}

\subsection{\codenamespace 扩展}
\label{sec:system:extensions}

允许网络内的各个节点支持功能的异构混合。
例如,某些节点可能希望将数据存储在数据库中或将事件发布到消息队列。
这些功能都是可选的,因为它们都不影响共识。
此类功能由节点加载的\emph{extensions}集确定,如\nemsetting{extensions-\{process\}}{extensions}中指定的那样。
大多数内置\codenamespace功能是使用此扩展模型构建的,以验证其可扩展性。

扩展是一个单独的动态链接库,它以以下形式开放单个接口
\footnote{扩展的格式取决于目标操作系统和所使用的编译器,因此所有主机应用程序和插件必须使用相同的编译器版本和选项来构建。}:

\begin{lstlisting}
extern "C" PLUGIN_API
void RegisterExtension(
	catapult::extensions::ProcessBootstrapper& 
	bootstrapper);
\end{lstlisting}

\class{ProcessBootstrapper} 类提供对扩展名自行初始化的完整\codenamespace配置和服务的访问。
提供此附加访问权限使扩展比插件更强大。
通过此类,扩展可以注册以下零个或多个:

\begin{enumerate}
	\item{服务(Services) -
		服务代表独立的行为。
		服务传递了一个表示可执行文件状态的对象,并且可以使用它来配置多种事物。
		服务尤其可以添加诊断计数器,定义API(诊断和非诊断API)并将任务添加到任务计划程序。
		它还可以创建从属服务,并将其生存期与托管可执行文件的生存期相关联。
		服务的功能几乎没有限制,这允许进行重大定制化开发。
	}
	\item{订阅(Subscriptions) -
		扩展可以订阅任何受支持的区块链事件。
		检测到更改时会引发事件。
		支持阻止,状态,未确认的事务和部分事务更改事件。
		事务处理完成时,将引发事务状态事件。
		发现远程节点时会引发节点事件。
	}
\end{enumerate}

除上述内容外,扩展还可以更复杂的方式配置节点。
例如,分机可以注册自定义网络时间源。
实际上,有一个专门的扩展程序可以根据\nemref{sec:timesync}中描述的算法建立时间源。
这是此扩展模型允许的高级别定制的一个示例。
要了解扩展所允许的全部扩展性,请参阅项目代码或开发人员文档。\nemtechdocsfootnote{}.

\subsection{服务器}

最简单的\codenamespace拓扑由单个服务器可执行文件组成。
网络和\codenamespace所需的事务插件节点操作员所需的扩展由服务器加载和初始化。

\codenamespace 将其所有数据存储在 \filepath{data} 目录(directory)中。
数据目录的内容如下:

\begin{enumerate}
	\item{区块版本目录(Block Versioned Directories) -
		这些目录包含专有格式的区块信息。
		每个已确认区块的二进制数据,事务记录和相关数据都存储在这些目录中。
		处理每个块时生成的语句 \nemrefparens{sec:blocks:receipts}也存储在此处以便快速访问。
		版本化目录的一个示例是 \filepath{00000},它包含第一组块。
	}
	\item{\filepath{audit} - Audit files created by the audit consumer \nemrefparens{sec:disruptor:commonConsumers} are stored in this directory.}
	\item{\filepath{spool} -
		Subscription notifications are written out to this directory.
		They are used as a message queue to pass messages from the server to the broker.
		They are also used by the recovery process to recover data in the case of an ungraceful termination.
	}
	\item{\filepath{state} -
		\codenamespace stores its proprietary storage files in this directory.
		\filepath{supplemental.dat} and files ending with \filepath{\_summary.dat} store summarized data.
		Files ending in \filepath{Cache.dat} store complete cache data.
	}
	\item{\filepath{statedb} - When \nemsetting{node}{enableCacheDatabaseStorage} is set, this directory will contain RocksDB files.}
	\item{\filepath{transfer\_message} -
		When \nemsetting{user}{enableDelegatedHarvestersAutoDetection} is set, this directory will contain extracted delegated harvesting requests for the current node.
	}
	\item{\filepath{commit\_step.dat} -
		This stores the most recent step of the commit process initiated by the server.
		This is primarily used for recovery purposes.
	}
	\item{\filepath{index.dat} - This is a counter that contains the number of blocks stored on disk.}
\end{enumerate}

\subsubsection{Cache Database}

The server can run with or without a cache database.
When \nemsetting{node}{enableCacheDatabaseStorage} is set, RocksDB is used to store cache data.
Verifiable state \nemrefparens{sec:blocks:statehash} requires a cache database and most network configurations are expected to have it enabled.

A cache database should only be \emph{disabled} when all of the following are true:

\begin{enumerate}
	\item{High rate of transactions per second is desired.}
	\item{Trustless verification of cache state is not important.}
	\item{Servers are configured with a large amount of RAM.}
\end{enumerate}

In this mode, all cache entries are always resident in memory.
On shutdown, cache data is written to disk across multiple flat files.
On startup, this data is read and used to populate the memory caches.

When a cache database is enabled, summarized cache data is written to disk across multiple flat files.
This summarized data is derivable from data stored in caches.
One example is the list all high-value accounts that have a balance of at least \nemsetting{network}{minHarvesterBalance}.
While this list can be generated by (re)inspecting all accounts stored in the account state cache, it is saved to and loaded from disk as an optimization.

\subsection{Broker}

The broker process allows more complex \codenamespace behaviors to be added without sacrificing parallelization.
Transaction Plugins required by the network and \codenamespace Extensions desired by the node operator are loaded and initialized by the broker.
Although the broker supports all features of Transaction Plugins, it only supports a subset of \codenamespace Extensions features.
For example, overriding the network time supplier in the broker is not supported.
Broker extensions are primarily intended to register subscribers and react to events forwarded to those subscribers.
Accordingly, it's expected that the server and broker have different extensions loaded.
Please refer to the project code or developer documentation for more details.

The broker monitors the \filepath{spool} directories for changes and forwards any event notifications to subscribers registered by loaded extensions.
Extensions register their subscribers to process these events.
For example, a database extension can read these events and use them to update a database to accurately reflect the global blockchain state.

\filepath{spool} directories function as one way message queues.
The server writes messages and the broker reads them.
There is no way for the broker to send messages to the server.
This decoupling is intentional and was done for performance reasons.

The server raises subscription events in the blockchain sync consumer \nemrefparens{sec:disruptor:blockConsumers} when it holds an exclusive lock to the blockchain data.
These operations are offloaded to the broker to prevent slow database operations when the server has an exclusive lock.
The server overhead is minimal because most of the data used by the broker is also required to recover data after an ungraceful server termination.

\subsection{Recovery}

The recovery process is used to repair the global blockchain state after an ungraceful server and/or broker termination.
Transaction Plugins required by the network and \codenamespace Extensions desired by the node operator are loaded and initialized by the recovery process.
When a broker is used, the recovery process must load the same extensions as the broker.

The specific recovery procedure depends on the process configuration and the value of the \filepath{commit\_step.dat} file.
Generally, if the server exited after state changes were flushed to disk, those changes will be reapplied.
The blockchain state will be the same as if the server had applied and committed those changes.
Otherwise, if the server exited before state changes were flushed to disk, pending changes will be discarded.
The blockchain state will be the same as if the server had never attempted to process those changes.

After the recovery process completes, the blockchain state should be indistinguishable from the state of a node that never terminated ungracefully.
\filepath{spool} directories are repaired and processed.
Block and cache data stored on disk are reconciled and updated.
Pending state changes, if applicable, are applied.
Other files indicating the presence of an ungraceful termination are updated or removed.

\subsection{Common Topologies}

Although a network can be composed of a large number of heterogeneous topologies, it is likely that most nodes will fall into one of three categories: Peer, API or Dual.
The same server process is used across all of these topologies.
The only difference is in what extensions each loads.

Peer nodes are lightweight nodes.
They have enough functionality to add security to the blockchain network, but little beyond that.
They can synchronize with other nodes and harvest new blocks.

API nodes are more heavyweight nodes.
They can synchronize with other nodes, but cannot harvest new blocks.
They support hosting bonded aggregate transactions and collecting cosignatures to complete them.
They require a broker process, which is configured to write data into a MongoDB database and propagate changes over public message queues to subscribers.
The REST API is dependent on both of these capabilities and is typically co-located with an Api node for performance reasons in order to minimize latency.

Dual nodes are simply a superset of Peer and API nodes.
They support all capabilities of both node types.
Since these nodes support all API node capabilities, they also require a broker.
