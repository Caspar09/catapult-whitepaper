\section{引言}
\label{sec:introduction}

\nemquote{
于灰烬中唤醒火焰,
于阴影中焕发光明;
折断的利刃将重露锋芒,
那无冕之人将重新称王。
}{J.R.R. 托尔金\footnote{J.R.R.托尔金,英国作家,诗人,著有《魔戒》系列。}}

基于区块链的去信任,高性能,分层体系结构的DLT协议-这些是影响\codename 研发的首要原则。
尽管考虑了其他DLT协议,包括DAG和dBFT,但很快就选择了区块链作为最符合去信任的理想协议。
任何节点都可以下载区块链的完整副本,并且可以随时独立进行验证。
具有足够收获能力的节点总是可以创建块,而无需依赖领导者。
这些选择必然会牺牲一些相对于其他协议的吞吐量,但它们似乎与比特币的哲学基础最一致。

作为聚焦去信任的一部分,在NEM的基础上添加了以下功能:
\begin{itemize}
    \item{可以在没有事务数据的情况下同步块头,同时允许验证链的完整性}
    \item{交易merkle树允许在区块内对交易是否存在进行加密证明}
    \item{收据增加了间接触发状态更改的透明度}
    \item{状态证明允许对区块链中的特定状态进行去信任验证}
\end{itemize}


在\codename中,有一个单一服务器可执行文件,可以通过加载不同的插件(用于事务支持)和扩展名(用于功能性)进行自定义。
(每个网络)有三种主要配置,但是通过启用或禁用特定扩展,可以进一步定制混合配置。

三种主要配置是:
\begin{enumerate}
\item Peer:Peer节点是网络的主心骨,他们会产出新的块。
\item API: API节点将数据存储在mongo数据库中,以便于查询,并且可以与NodeJS REST服务器结合使用。
\item Dual: Dual节点同时具有Peer节点和Api节点的功能。
\end{enumerate}


一个强大的网络通常将具有大量的Peer节点和足够的API节点来支持传入的客户端请求。
允许节点的组成根据实际需求动态变化,应该网络的资源全局更加优化。

相对于典型的区块链协议,以干扰器(Disruptor)模式为基础的核心区块和事务流水线-并尽可能使用并行处理-可以实现每秒较高的事务速率。


NIS1是值得记录在区块链领域的技术,\codenamespace是其更有价值的演变。
这不是结束,而仅仅是新的开始。
我们还有更多工作要做。

\subsection{变种}

\codenamespace 支持其主要哈希算法的编译时间替换,该算法从输入数据中生成32字节的值。
为简洁起见,本文档的其余部分将在假定其默认设置(SHA3)的情况下引用此哈希。
当前,支持以下哈希算法:

\begin{enumerate}
    \item{SHA3 (default): SHA3是默认模式,建议新的链采用。}
    \item{Keccak: 提供此模式是为了与诸如NIS1之类的旧式链兼容,以便保留私钥:公钥:地址之间的映射。}
\end{enumerate}
